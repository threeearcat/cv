%%%%%%%%%%%%%%%%%%%%%%%%%%%%%%%%%%%%%%%%%
% Medium Length Professional CV
% LaTeX Template
% Version 2.0 (8/5/13)
%
% This template has been downloaded from:
% http://www.LaTeXTemplates.com
%
% Original author:
% Trey Hunner (http://www.treyhunner.com/)
%
% Important note:
% This template requires the resume.cls file to be in the same directory as the
% .tex file. The resume.cls file provides the resume style used for structuring the
% document.
%
%%%%%%%%%%%%%%%%%%%%%%%%%%%%%%%%%%%%%%%%%

%----------------------------------------------------------------------------------------
%	PACKAGES AND OTHER DOCUMENT CONFIGURATIONS
%----------------------------------------------------------------------------------------

\documentclass{resume} % Use the custom resume.cls style

\usepackage[left=0.75in,top=0.6in,right=0.75in,bottom=0.6in]{geometry} % Document margins
\usepackage{kotex}
\usepackage{tran_paul_le_cv}
\usepackage[colorlinks=true]{hyperref}

\name{Dae R. Jeong (정대룡)} % Your name
\address{E3-1 4427, KAIST, Daejeon, Republic of Korea 34141}
\address{dae.r.jeong@kaist.ac.kr \\ threeearcat@gmail.com (personal)} % Your phone number and email

\bibliography{references, conf}

\begin{document}

\begin{rSection}{Research Interests}

\end{rSection}

%----------------------------------------------------------------------------------------
%	EDUCATION SECTION
% ----------------------------------------------------------------------------------------

\begin{rSection}{Education}

B.S. in School of Computing, KAIST \hfill {\em Mar. 2010 - Feb. 2014}

M.S. in School of Computing KAIST \hfill {\em Mar. 2014 - Feb. 2016} \\
\textit{Advisor: Insik Shin}

Ph.D in School of Computing, KAIST \hfill {\em Mar. 2016 - Feb. 2023} \\
\textit{Advisor: Insik Shin} \\
\textit{Thesis: Finding and Diagnosing Concurrency Bugs in a Kernel through Systematic Instruction Scheduling}

\begin{itemize}
  \item{Outstanding  dissertation award}
\end{itemize}

\end{rSection}

%----------------------------------------------------------------------------------------
%	WORK EXPERIENCE SECTION
%----------------------------------------------------------------------------------------

\begin{rSection}{Experience}

  \denseouterlist {
  \item{\textbf{Teaching Assistant}}
    \entrylabeledextra{-- }{Operating System and Lab (CS330) \hfill {Spring 2018}}
    \entrylabeledextra{-- }{Operating System (CS530) \hfill {Fall 2017}}
    \entrylabeledextra{-- }{Operating System and Lab (CS330) \hfill {Spring 2016}}
    \entrylabeledextra{-- }{Operating System and Lab (CS330) \hfill {Spring 2015}}
    \entrylabeledextra{-- }{Operating System and Lab (CS330) \hfill {Spring 2014}}

  \normalsize \item{\textbf{Head Teaching Assistant}}
    \entrylabeledextra{-- }{Operating System and Lab (CS330) \hfill {Fall 2019}}
    \entrylabeledextra{-- }{Operating System and Lab (CS330) \hfill {Spring 2017}}
  }

\end{rSection}

\addtocategory{intlconf}{
  aitia,
  hfl,
  fluid,
  mobileplus,
  razzer,
  choi2019light,
  mobileplusmdm,
  segfuzz,
  mixmax,
}

\addtocategory{misc}{
  mobap-cpu,
  mobap,
  mobileplusposter,
  gpgpu,
}

%----------------------------------------------------------------------------------------
%	EDUCATION SECTION
% ----------------------------------------------------------------------------------------

\begin{rSection}{Publications}
\begin{publications}
  \smallskip
  \printbib{intlconf}
  \bigskip
  \printbib{misc}
\end{publications}
\end{rSection}

\begin{rSection}{Open Source Contribution}
  \denseouterlist{
  \item \textbf{SegFuzz}: A kernel fuzzer utilizing interleaving coverage to discover concurrency bugs\\
    \url{https://github.com/casys-kaist/segfuzz}
  \item \textbf{HFL}: A hybrid kernel fuzzer combining symbolic execution and fuzzing\\
    \url{https://bitbucket.org/anonyk/hfl-release/src/master/}
  \item \textbf{Linux}: Reported and fixed concurrency bugs in various subsystems, Contributor\\
    \url{https://git.kernel.org/pub/scm/linux/kernel/git/torvalds/linux.git/}
  \item \textbf{Razzer}: A kernel fuzzer focusing on concurrency bugs\\
    \url{https://github.com/compsec-snu/razzer}
  \item \textbf{QEMU}: Reported some bugs (with other students)\\
    \url{https://qemu.org/}
  \item \textbf{Android Open Source Project (AOSP)}: Reported some bugs (with other students)\\
    \url{https://source.android.com/}
  }
\end{rSection}
% ----------------------------------------------------------------------------------------
%	HONORS and AWARDS SECTION
% ----------------------------------------------------------------------------------------

\begin{rSection}{Honors and Awards}
  \denseouterlist{

  \item \textbf{Program Directors Award}, 2023\\
    \small Samsung Global Technology Symposium

  \normalsize \item \textbf{Outstanding Dissertation Award}, 2023\\
    \small School of Computing, KAIST\\
    \small \textit{Finding and Diagnosing Concurrency Bugs in a Kernel through Systematic Instruction Scheduling}

    \normalsize \item \textbf{Best Paper Award}, 2021\\
    \small Korea Institute of Information Scientists and Engineers (한국정보과학회)\\
    \small \textit{{MoBaP}: Mobile Battery Prediction Framework for Video Streaming},

    \normalsize \item \textbf{Best Paper Award}, 2019\\
    \small ACM International Conference on Mobile Computing and Networking (MobiCom)\\
    \small \textit{FLUID: Multi-device Mobile Platform for Flexible User Interface Distribution}

    \normalsize \item \textbf{Naver Ph.D Fellowship Award}, 2019

    \normalsize \item \textbf{Second prize (우수상)}, 2015\\
    \small E*5 LabStartup KAIST\\
    \small \textit{Team LeviOsa}

    \normalsize \item \textbf{Undergraduate Student Best Paper Award, 2015}\\
    \small Korea Institute of Information Scientists and Engineers (한국정보과학회)\\
    \small \textit{GPGPU Parallelization Techniques for Redundancy Elimination Algorithm}
  }
\end{rSection}

\begin{rSection}{Patents}
  \denseouterlist{
  \item 무인비행체 조종 방법, 이를 구현하기 위한 프로그램이 저장된 기록매체 및 이를 구현하기 위해 매체에 저장된 컴퓨터프로그램, 1020180052585 (2018.05.08)\\
  \scriptsize METHOD FOR CONTROLING UNMANNED FLYING OBJECT AND RECORDING MEDIUM STORING PROGRAM FOR EXECUTING THE SAME, AND RECORDING MEDIUM STORING PROGRAM FOR EXECUTING THE SAME

  \normalsize \item 원통좌표계 기반 무인이동체 조종 방법, 이를 구현하기 위한 프로그램이 저장된 기록매체 및 이를 구현하기 위해 매체에 저장된 컴퓨터프로그램, 1020180052598 (2018.05.08)\\
  \scriptsize METHOD FOR CONTROLING UNMANNED MOVING OBJECT BASED ON CYLINDRICAL COORDINATE SYSTEM AND RECORDING MEDIUM STORING PROGRAM FOR EXECUTING THE SAME, AND COMPUTER PROGRAOM STORED IN RECORDING MEDIUM FOR EXECUTING THE SAME

  \normalsize \item 어플리케이션 수행에 있어서 모바일 기기 간에 기능을 분배하는 방법, 1020170089910 (2017.07.14)\\
  \scriptsize METHOD FOR CROSS-DEVICE FUNCTIONALITY SHARING

  \normalsize \item 무인이동체 조종 방법, 이를 구현하기 위한 프로그램이 저장된 기록매체 및 이를 구현하기 위해 매체에 저장된 컴퓨터프로그램, 1017518640000 (2017.06.22)\\
  \scriptsize SMART DEVICE FOR CONTROLING UNMANNED MOVING OBJECT AND METHOD FOR CONTROLING UNMANNED MOVING OBJECT AND RECORDING MEDIUM STORING PROGRAM FOR EXECUTING THE SAME, AND RECORDING MEDIUM STORING PROGRAM FOR EXECUTING THE SAME
}

\end{rSection}

%----------------------------------------------------------------------------------------
%	TECHNICAL STRENGTHS SECTION
%----------------------------------------------------------------------------------------

\begin{rSection}{Technical Strengths}

\begin{tabular}{ @{} >{\bfseries}l @{\hspace{6ex}} l }
Computer Languages & Prolog, Haskell, AWK, Erlang, Scheme, ML \\
Protocols \& APIs & XML, JSON, SOAP, REST \\
Databases & MySQL, PostgreSQL, Microsoft SQL \\
Tools & SVN, Vim, Emacs
\end{tabular}

\end{rSection}

%----------------------------------------------------------------------------------------
%	EXAMPLE SECTION
%----------------------------------------------------------------------------------------

%\begin{rSection}{Section Name}

%Section content\ldots

%\end{rSection}

%----------------------------------------------------------------------------------------

\end{document}
